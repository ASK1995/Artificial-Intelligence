\subsection{Beam Search}
\begin{algorithm}[!h]
\caption{Beam Search}
\label{Beam}
\begin{algorithmic}	
\STATE function Astar(state, beamWidth, heuristicType):
\STATE Q:=heapify(state)
\IF{state is goal state}
	\RETURN \textit{solutionPath(state)}
\ENDIF
\IF{disk is not last moved}
	\STATE successors = move disks column top disks
	\STATE cost = \textit{heuristic(successors)}
\ENDIF 
\FOR{s in successors}
\IF{s in \textit{path.keys()}}
	\STATE delete s
\ELSIF{s in \textit{path.keys()}}
	\STATE keep s with min cost
\ELSE
	\STATE Q.push(s) 
\ENDIF
\ENDFOR
	
\IF{Q is empty}
	\RETURN failure
\ELSE
	\FOR{i in 1:beamWidth}
		\STATE mQ.append(Q.pop())
		\STATE path.Add(s: s.parent)
	\ENDFOR
	\RETURN mQ
\ENDIF
	\end{algorithmic}
\end{algorithm}

Beam search is described by Algorithm~\ref{Beam}.  
Beam search uses the same heuristics as Astar; however, instead of maintaining a frontier across tree depths and between expansions, it pushes to a size-limited sorted heap and pops the entire queue after every turn.
The size of the frontier is limited by the beam width parameter.
Since the queue is emptied and repopulated every turn, all frontier nodes exist at the same depth.
Beam search maintains a map creating links from children to parent nodes as they are expanded.  
This mapping provides a way to reclaim the path and a set of all previously visited nodes which we use to exclude current child nodes from the queue.