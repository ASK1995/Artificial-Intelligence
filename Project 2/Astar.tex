\subsection{Astar}
Algorithm~\ref{astar} describes Astar search, a type of best-first search which uses a heuristic to gauge the distance to the goal, $h$, and adds it to the path cost, $g$, to choose the next node to expand, that is, the direction that seems most likely to reach the goal first.
Astar uses a priority queue implemented on a heap to sort the best potential nodes on the frontier.
The heap stores tuples for each frontier node which take the form $(heuristic cost ~ [g + h], g, [frontierState.parent, frontierState])$.
When a node is chosen, it is removed from the queue and added to the \textit{path} where the node is the key an the parent is the value (the parent is already a key in the map pointing to it's parent).
When we find the goal, we travel up the map collecting the path using the path reconstruction function given in Algorithm~\ref{alg:reconstructpath}.

\begin{algorithm}[!h]
\caption{Astar Search}
\label{astar}
\begin{algorithmic}
\STATE function Astar(state, heuristicType):
\STATE Q:=heapify(state)
\IF{state is goal state}
	\RETURN \textit{solutionPath(state)}
\ENDIF
\IF{disk is not last moved}
    \STATE successors = move disks column top disks
    \STATE cost = \textit{heuristic(successors)}
\ENDIF 
\FOR{s in successors}
    \IF{s in Q}
        \STATE keep s with min cost
    \ELSIF{s in \textit{path.keys()}}
        \STATE delete s
    \ELSE
        \STATE Q.push(s)
    \ENDIF
\ENDFOR

\IF{Q is empty}
	\RETURN failure
\ELSE
	\STATE minQ := Q.pop()
    \STATE path.Add(minQ: minQ.parent)
    \RETURN minQ
\ENDIF
\end{algorithmic}
\end{algorithm}
